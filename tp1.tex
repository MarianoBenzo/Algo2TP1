\documentclass[10pt, a4paper]{article}
\usepackage[paper=a4paper, left=1.5cm, right=1.5cm, bottom=1.5cm, top=3.5cm]{geometry}
\usepackage[latin1]{inputenc}
\usepackage[T1]{fontenc}
\usepackage[spanish]{babel}
\usepackage{indentfirst}
\usepackage{fancyhdr}
\usepackage{latexsym}
\usepackage{lastpage}
\usepackage{aed2-symb,aed2-itef,aed2-tad, caratula}
\usepackage[colorlinks=true, linkcolor=blue]{hyperref}
\usepackage{calc}

\newcommand{\f}[1]{\text{#1}}
\renewcommand{\paratodo}[2]{$\forall~#2$:#1}

\sloppy

%  \hypersetup{%
%   % Para que el PDF se abra a p�gina completa.
%   pdfstartview= {FitH \hypercalcbp{\paperheight-\topmargin-1in-\headheight}},
%   pdfauthor={C�tedra de Algoritmos y Estructuras de Datos II - DC - UBA},
%   pdfkeywords={TADs b�sicos},
%   pdfsubject={Tipos abstractos de datos b�sicos}
%  }

\parskip=5pt % 10pt es el tama�o de fuente

% Pongo en 0 la distancia extra entre �temes.
\let\olditemize\itemize
\def\itemize{\olditemize\itemsep=0pt}

% Acomodo fancyhdr.
\pagestyle{fancy}
\thispagestyle{fancy}
\addtolength{\headheight}{1pt}
\lhead{Algoritmos y Estructuras de Datos II}
\rhead{$1^{\mathrm{er}}$ cuatrimestre de 2015}
\cfoot{\thepage /\pageref{LastPage}}
\renewcommand{\footrulewidth}{0.4pt}

\author{Algoritmos y Estructuras de Datos II, DC, UBA.}
\date{}
\title{Tipos abstractos de datos b�sicos}

\begin{document}


% Estos comandos deben ir antes del \maketitle
\materia{Algoritmos y Estructuras de Datos II} % obligatorio
\submateria{Primer Cuatrimestre de 2015} % opcional
\titulo{Trabajo Pr�ctico 1} % obligatorio
\subtitulo{Especificaci�n} % opcional


\integrante{INTEGRANTE, 1}{123/12}{1@gmail.com} % obligatorio 
\integrante{INTEGRANTE, 2}{123/12}{2@gmail.com} % obligatorio 
\integrante{INTEGRANTE, 3}{123/12}{3@gmail.com} % obligatorio 
\integrante{INTEGRANTE, 4}{123/12}{4@gmail.com} % obligatorio 
%Pagina de titulo e indice
\thispagestyle{empty}

\maketitle
\tableofcontents

\newpage


\section{TAD \tadNombre{Dato}}
\begin{tad}{\tadNombre{Dato}}
\tadGeneros{dato}

\tadIgualdadObservacional{d}{d'}{dato}{$(EsNat?(d)=1\land EsNat?(d)=EsNat?(d'))\impluego DNat(d)=DNat(d') \lor (EsNat?(d)= 0 \land EsNat?(d')=EsNat?(d')) \impluego DString(d)=DString(d')$}

\tadUsa{Bool, Nat, String}

\tadExporta{\tadNombre{}}

\tadObservadores

\tadOperacion{EsNat?}{dato}{bool}{}
\tadOperacion{DNat}{dato/d}{nat}{$EsNat?(d)$}
\tadOperacion{DString}{dato/d}{string}{$\neg EsNat?(d)$}


\tadGeneradores

\tadOperacion{NDat}{bool, nat, string}{dato}{}


\tadAxiomas[]


\tadAxioma{EsNat?(NDato(b,n,s))}{$b$}
\tadAxioma{DNat(NDato(b,n,s))}{$n$}
\tadAxioma{DString(NDato(b,n,s))}{$s$}



\end{tad}


\section{TAD \tadNombre{Registro}}
\begin{tad}{\tadNombre{Registro}}
\tadGeneros{reg}

\tadUsa{Lista, Conjunto, Nat, Bool, \tadNombre{Dato}}

\tadExporta{\tadNombre{}}

\tadIgualdadObservacional{r}{r'}{reg}{$Campos(r)=Campos(r')\land(\forall c: String)tupla(c,bool)\in Campos(r) \impluego Dato?(r)=Dato(r'))$}

\tadObservadores

\tadOperacion{Campos}{reg}{$conj(tupla(string, bool))$}{}
\tadOperacion{Dato?}{reg/r, String/c}{Dato}{$tupla(c,True) \in Campos(r) \lor tupla(c,False) \in Campos(r)$}

\tadGeneradores

\tadOperacion{NReg}{}{reg}{}
\tadOperacion{AgCampo}{reg/r, String/c, Bool/b, Dato/d}{reg}{$EsNat?(d)\equiv b$}

\tadOtrasOperaciones

\tadOperacion{JRegs}{reg/r, reg/s}{reg}{}
\tadOperacion{tgregs}{conj(campo)/c, reg/r, dicc(campo,dato)/d}{reg}{($\forall$ $c_1$:campo) $(c_1 \in c)\Rightarrow (c_1 \in Campos(r) \lor c_1 \in claves (d)) $}

\tadAxiomas[]


\tadAxioma{Campos(NReg)}{$\emptyset$}
\tadAxioma{Campos(AgCampo(r, c, b, d))}{$Ag(tupla(c, b))$}
\tadAxioma{Dato?(AgCampo(r, c, b, d), c')}{\IF $c=c'$ THEN
												$d$
												ELSE
												$Dato?(r)$
												FI}
\tadAxioma{Campos(JRegs(r, s))}{$Campos(r) \cup  Campos(s)$}
\tadAxioma{Dato?(JRegs(r, s), c)}{\IF $c \in Campos(r)$ THEN
										$Dato?(r, c)$
										ELSE
										$Dato?(s, c)$
										FI}
\tadAxioma{tgregs(c,r,d)}{\IF $c= \emptyset$ THEN Nreg ELSE {\IF $DameUno (c) \in Campos (r)$ THEN $AgCampo(tgregs(SinUno(c),r,d), DameUno(c), Dato? (r,Dameuno(c)))$ ELSE $AgCampo(tgregs(SinUno(c),r,d), DameUno(c), Obtener (DameUno(c),d))$ FI  } FI }

\end{tad}


\section{TAD \tadNombre{Tabla}}
\begin{tad}{\tadNombre{Tabla}}
\tadGeneros{tab}

\tadUsa{}

\tadExporta{\tadNombre{}}

\tadIgualdadObservacional{t}{t'}{tab}{$Campos(t)=Campos(t') \land Claves(t)=Claves(t')$}

\tadObservadores


\tadOperacion{CamposT}{tab}{$conj(campo)$}{}
\tadOperacion{Claves}{tab}{$conj(campo)$}{}


\tadGeneradores

\tadOperacion{NTab}{conj(campo)/cp,conj(campo)/cl}{tab}{$($\paratodo{campo}{c}$) c \in cl \implies c \in cp$}


\tadAxiomas[]

\tadAxioma{Campos(NTab(cp, cl))}{$cp$}
\tadAxioma{Claves(NTab(cp, cl))}{$cl$}





\end{tad}



\section{TAD \tadNombre{BaseDeDatos}}
\begin{tad}{\tadNombre{BaseDeDatos}}
\tadGeneros{bds}


\tadUsa{Bool, Nat, String, Conjunto, Dicc(clave,significado), Tupla(),Dato, Tabla, Registro}

\tadExporta{}

\tadObservadores


\tadOperacion{Tablas}{bds}{$conj(tab)$}{}
\tadOperacion{RegistrosT}{bds/b, tab/t}{$conj(reg)$}{$t \in Tablas(b)$}
\tadOperacion{Joins}{bds}{$dicc(tupla(conj(tab),campo),tab))$}{}
\tadOperacion{RegistrosJ}{bds/b, tupla(conj(tab),campo)/tj}{$conj(reg)$}{$tj \in Claves(Joins(b)) $}
\tadOperacion{Triggers}{bds/b, tab/t}{$dicc(tab/t_2 , dicc(campo/c, dato)$}{$t \in Tablas(b)$}
\tadOperacion{Modificaciones}{bds/b, tab/t}{$Nat$}{$t \in Tablas(b)$}

\tadGeneradores

\tadOperacion{NuevaBase}{conj(tab)}{bds}{}
\tadOperacion{AgTab}{bds/b, tab/t}{bds}{$t \notin Tablas(bds)$}
\tadOperacion{AgReg}{bds/b, tab/t, reg/r}{bds}{$CamposT(t)=Campos(r) \land (($\paratodo{reg}{r'}, \paratodo{campo}{c}$)(r' \in \Pi_2(t) \land c \in claves(t) \impluego (Dato?(r, c) \neq Dato?(r', c)))$}
\tadOperacion{ElimReg}{bds/b, tab/t, campo/c, dato/d}{bds}{$c \in CamposT(t) \land EsNat?(d)=\Pi_2 (c) $}
\tadOperacion{CrearJoin}{bds/b, tab/t1, tab/t2, campo/cl}{bds}{$t1,t2 \in Tablas(bds) \land cl \in campos(t1) \land cl \in campos(t2) \land tupla(\{t1,t2\},cl) \notin Claves(Joins(b))$}
\tadOperacion{EliminarJoin}{bds/b, tab/t1, tab/t2, campo/cl}{bds}{$tupla(\{t1,t2\},cl) \in Claves(Joins(b))$}
\tadOperacion{AgTrigger}{bds/b, tab/t1, tab/t2, dicc(campo, dato)/default}{bds}{$(claves(t2) \subseteq claves(t1)) \land (\forall c:campo (c \in claves(default) \Leftrightarrow (c \in campos(t2) \land c \notin campos(t1))))$}
\tadOperacion{EliminarTrigger}{bds/b, tab/t1, tab/t2}{bds}{}

\tadOtrasOperaciones

\tadOperacion{ERR}{conj(reg), campo, dato}{conj(reg)}{}

%% otras operaciones usadas en join %%
\tadOperacion{JoinRegistros}{conj(reg),conj(reg),campo}{conj(reg)}{}
\tadOperacion{DameRegCon}{conj(reg),dato,campo}{conj(reg)}{}
\tadOperacion{DameRegSin}{conj(reg),dato,campo}{conj(reg)}{}

%%operaciones de más modificada%%
\tadOperacion{TMK}{conj(tab)/c,bds/b}{nat}{$c \subseteq Tablas(b)$}
\tadOperacion{TM}{conj(tab)/c,bds/b}{conj(tab)}{$c \subseteq Tablas(b)$}
\tadOperacion{MasModificada}{bds/b}{tab}{}

\tadAxiomas[]

\tadAxioma{Tablas(NuevaBase(ct))}{$ct$}
\tadAxioma{Tablas(Agtab(b, t))}{$Ag(t, Tablas(b))$}
\tadAxioma{Tablas(AgReg(b, t, r))}{$Tablas(b)$}
\tadAxioma{Tablas(ElimReg(b, t, r))}{$Tablas(b)$}
\tadAxioma{Tablas(CrearJoin(b, t1, t2, c))}{$Tablas(b)$}
\tadAxioma{Tablas(AgTrigger(b, t1, t2, d))}{$Tablas(b)$}


\tadAxioma{RegistrosT(AgTab(b,t1),t)}{\IF $t = t1$ THEN
												$\emptyset$
											ELSE
												$RegistrosT(b,t)$
											FI}
\tadAxioma{RegistrosT(AgReg(b, t1, r),t)}{\IF $t = t1$ THEN
												$Ag(r, RegistrosT(b,t))$
											ELSE {\IF $t \in claves(Triggers(b,t1)$ THEN $Ag(tgregs(campos(t), r, Obtener(t, Triggers(b, t1))), \\ RegistrosT (b,t))$ ELSE $RegistrosT(b,t)$ FI }									
											FI}
\tadAxioma{RegistrosT(ElimReg(b, t1, c, d),t)}{$ERR(RegistrosT(b,t), c, d)$}
\tadAxioma{RegistrosT(CrearJoin(b,t1,t2,c))}{$RegistrosT(b,t)$}
\tadAxioma{RegistrosT(AgTrigger(b,t1,t2,d))}{$RegistrosT(b,t)$}


\tadAxioma{ERR(rs, c, d)}{\IF $rs = \emptyset$ THEN
												$\emptyset$
												ELSE
													{\IF $d = Dato?(DameUno(rs), c)$ THEN
													$ERR(SinUno(rs), c, d)$
													ELSE
													$Ag(DameUno(rs),ERR(SinUno(rs),c ,d)$
													FI}
												FI}


\tadAxioma{Joins(NuevaBase(ct))}{$vacío$}
\tadAxioma{Joins(AgTab(bds, t))}{${Joins(bds)}$}
\tadAxioma{Joins(CrearJoin(bds, t1, t2, c))}{$Definir(tupla({t1,t2},c),Ntab((CamposT(t1) \cup CamposT(t2)-c),\\(ClavesT(t1) \cup ClavesT(t2)-c)),Joins(bds))$}
\tadAxioma{Joins(EliminarJoin(bds, t1, t2, c))}{$Borrar(tupla({t1,t2},c),Joins(bds))$}
\tadAxioma{Joins(AgReg(bds, t, r))}{$Joins(bds)$}
\tadAxioma{Joins(ElimReg(bds, t, c, d))}{$Joins(bds)$}
\tadAxioma{Joins(AgTrigger(bds, t, c, d))}{$Joins(bds)$}


\tadAxioma{RegistrosJ(AgTab(bds, t), tj)}{${RegistrosJ(bds,tj)}$}
\tadAxioma{RegistrosJ(CrearJoin(bds, t1, t2, c), tj)}{
	\IF $t1 \in \pi_1(tj) \land t2 \in \pi_1(tj) \land c=\pi_2(tj)$ THEN
	$JoinRegistros(RegistrosT(t1), RegistrosT(t2), c)$
	ELSE
	$RegistrosJ(bds, tj)$
	FI
	}
\tadAxioma{RegistrosJ(EliminarJoin(bds, t1, t2, c), tj)}{$RegistrosJ(bds, tj)$}
\tadAxioma{RegistrosJ(AgReg(bds, t, r), tj)}{\IF $t \notin \pi_1(tj)$ THEN
											$RegistrosJ(bds,tj)$
											ELSE
												{\IF $DameRegCon(RegistrosT(\pi_1(tj)-t), Dato?(r, \pi_2(tj)), \pi_2(tj)) \neq \emptyset$ THEN
													$Ag(Jreg(DameUno(DameRegCon(RegistrosT(\pi_1(tj) - t), \\ Dato?(r, \pi_2(tj)), \pi_2(tj))),r), RegistrosJ(bds,tj))$
													ELSE
													$RegistrosJ(bds,tj)$
												FI}
											FI}
		

\tadAxioma{RegistrosJ(ElimReg(bds, t, c, d), tj)}{\IF $t \in \pi_1(tj)$ THEN
													$DameRegSin(RegistrosJ(bds,tj),d,c)$
													ELSE
													$RegistrosJ(bds,tj)$
													FI}

\tadAxioma{RegistrosJ(AgTrigger(bds, t1, t2,d), tj)}{${RegistrosJ(bds,tj)}$}

\tadAxioma{Triggers(NuevaBase(c) ,t)}{$\emptyset$}
\tadAxioma{Triggers(AgTab(b,t1) ,t)}{$Triggers(b,t)$}
\tadAxioma{Triggers(AgReg(b,t1,r) ,t)}{$Triggers(b,t)$}
\tadAxioma{Triggers(EliminarReg(b,t1,c,d) ,t)}{$Triggers(b,t)$}
\tadAxioma{Triggers(CrearJoin(b,t1,t2,cl) ,t)}{$Triggers(b,t)$}
\tadAxioma{Triggers(EliminarJoin(b,t1,t2,cl) ,t)}{$Triggers(b,t)$}
\tadAxioma{Triggers(AgTrigger(b,t1,t2,d) ,t)}{\IF $t=t1$ THEN $definir (Triggers(b,t), t2,d)$ ELSE $Triggers(b,t)$ FI}
\tadAxioma{Triggers(EliminarTrigger(b,t1,t2) ,t)}{\IF $t=t1$ THEN $borrar(Triggers(b,t), t2)$ ELSE $Triggers(b,t)$ FI}

\tadAxioma{Modificaciones(NuevaBase(c) ,t)}{$0$}
\tadAxioma{Modificaciones(AgTab(b,t1) ,t)}{\IF $t=t1$ THEN $0$ ELSE $Modificaciones(b,t)$ FI}
\tadAxioma{Modificaciones(AgReg(b,t1,r) ,t)}{\IF $t=t1 \lor t \in claves(Triggers(b,t1))$ THEN $Modificaciones(b,t)+1$  ELSE $Modificaciones(b,t)$ FI}
\tadAxioma{Modificaciones(EliminarReg(b,t1,c,d) ,t)}{\IF $t=t1$ THEN $Modificaciones(b,t)+1$  ELSE $Modificaciones(b,t)$ FI}
\tadAxioma{Modificaciones(CrearJoin(b,t1,t2,cl) ,t)}{$Modificaciones(b,t)$}
\tadAxioma{Modificaciones(EliminarJoin(b,t1,t2,cl) ,t)}{$Modificaciones(b,t)$}
\tadAxioma{Modificaciones(AgTrigger(b,t1,t2,d) ,t)}{$Modificaciones(b,t)$}
\tadAxioma{Modificaciones(EliminarTrigger(b,t1,t2) ,t)}{$Modificaciones(b,t)$}

\tadAxioma{JoinRegistros(r1,r2,c)}{
		\IF $r1 = \emptyset \lor r2 = \emptyset$ THEN
				$\emptyset$
			ELSE
				{\IF $DameRegCon(r1,Dato?(DameUno(r2),c),c) \neq \emptyset$ THEN
						$Ag(JRegs(DameUno(r2),\\ DameUno(DameRegCon(r1,Dato?(DameUno(r2),c)))), \\ JoinRegistros(r1, SinUno(r2),c))$
					ELSE
						$JoinRegistros(r1, SinUno(r2),c)$
				FI}
			FI}


\tadAxioma{DameRegCon(r,d,c)}{
		\IF $r = \emptyset$ THEN
				$\emptyset$
			ELSE
				{\IF $Dato?(DameUno(r),c)=d$ THEN
						${DameUno(r)}$
					ELSE
						$DameRegCon(SinUno(r),d,c)$
				FI}
			FI}


\tadAxioma{DameRegSin(r,d,c)}{
		\IF $r = \emptyset$ THEN
				$\emptyset$
			ELSE
				{\IF $Dato?(DameUno(r),c)=d$ THEN
						$DameRegSin(SinUno(r),d,c)$
					ELSE
						$Ag(DameUno(r),DameRegSin(SinUno(r),d,c))$
				FI}
			FI}
\tadAxioma{MasModificada(b)}{DameUno(TM(tablas(b),b))}
\tadAxioma{TM(c,b)}{\IF $c= \emptyset$  THEN $\emptyset$ ELSE {\IF $Modificaciones(DameUno(c),b)=TMK(c,b) $THEN Ag(DameUno(c), TM(SinUno(c),b)) ELSE TM(SinUno(c),b) FI} FI}
\tadAxioma{TMK(c,b)}{\IF $c= \emptyset$ THEN $0$ ELSE {\IF $Modificaciones(DameUno(c),b)  > TMK(SinUno(c),b)$ THEN Modificaciones(DameUno(c),b) ELSE TMK(SinUno(c),b) FI} FI}
			
			
\end{tad}

\end{document}
\grid
\grid
